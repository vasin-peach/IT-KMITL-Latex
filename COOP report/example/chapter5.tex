\chapter{สรุปผล}
\label{chapter:conclusion}

\section{ผลการทดลองและการแก้เปัญหาในอนาคต}
% ในการทดลองนี้ เราได้เสนอวิธีการตรวจหาข้อความบนภาพมังงะด้วยเทคนิค SWT ร่วมกับการใช้ SVM และลักษณะเด่น (Feature) HOG ในการลด False Positive ที่เกิดขึ้น การทดลองของเราดำเนินการบนชุดข้อมูล Manga109 ซึ่งเป็นชุดข้อมูลภาพมังงะที่ถูกระบุ Annotation ที่เกี่ยวข้องมาเรียบร้อยแล้ว วิธีการของเรานั้นสามารถทำงานได้ผลลัพธ์ F-measure ที่ดีที่สุดในการเปรียบเทียบกับวิธี Baseline และวิธีการอื่น ๆ ที่เกี่ยวข้องรวมถึงวิธีการที่ใช้ Deep Learning เป็นส่วนประกอบในการคัดแยกอักษรหรือข้อความจากวัตถุอื่น ๆ ในภาพ ถึงแม้ว่างานของเรานั้นได้ทดสอบบนการ์ตูนญี่ปุ่นสามารถทำงานได้เป็นผลดีเยี่ยม แต่วิธีการของเรานั้นยังต้องมีการพัฒนาและค้นคว้าเพิ่มเติมเพื่อปรับปรุงประสิทธิภาพและทำให้สามารถใช้งานร่วมกับภาษาอื่น ๆ ได้
ในการทดลองนี้ เราได้เสนอกรอบการทำงานของระบบแนะนำตำแหน่งงานโดยใช้เทคนิคการกรองแบบเนื้อหา โดยการทำงานถูกแบ่งออกเป็นสองส่วนคือ ส่วนของการแบ่งกลุ่มยูสเซอร์ในหมวดฟิลด์ไอทีต่าง ๆ ที่กำหนดไว้โดยใช้โมเดล SVM ซึ่งจากผลการทดลองโมเดลมีความแม่นยำในการแบ่งกลุ่มโปรไฟล์เพียง 77\% ซึ่งถือว่าเกือบพอใช้ได้ และในส่วนที่สองจะเป็นการจับคู่ระหว่างโปรไฟล์ที่ทราบกลุ่มฟิลด์แล้วกับตำแหน่งงานในฟิลด์นั้น ๆ โดยประสิทธิภาพจะขึ้นอยู่กับการแบ่งกลุ่มของโมเดลการแบ่งกลุ่มโปรไฟล์ว่ามีความแม่นยำเพียงใด

ทิศทางในอนาคตของงานเราจะมุ่งเน้นไปที่การประเมินผลที่มีความละเอียดยิ่งขึ้นโดยพิจารณาการรวบรวมข้อมูลให้มีความซื่อตรงมากที่สุด และจำนวนข้อมูลที่มากกว่าเดิม รวมถึงประเมินที่กลุ่มที่ครอบคลุมมากขึ้นซึ่งหมายถึงตำแหน่งงานที่ผู้หางานจะได้รับหลากหลายมากขึ้น
และในเฟสต่อไปหลังจากที่ปรับปรุงโมเดลให้มีประสิทธิภาพตามที่คาดหวังไว้แล้ว จะดำเนินการสร้างเว็บแอพพลิเคชั่นที่รองรับระบบนี้อย่างเต็มรูปแบบ และเปิดใช้งานในอนาคต

\section{ปัญหาที่เกิดขึ้น}
เนื่องจากความกว้างขวางของฟิลด์ไอที ทำให้ในการดึงข้อมูลจำเป็นต้องมีความรอบคอบใน keyword ที่ใช้ในการค้นหามิเช่นนั้นจะทำให้การความคาดเคลื่อนเป็นจำนวนมากกับโมเดล และในส่วนของโปรไฟล์นั้นอคติที่เกิดขึ้นกับโปรไฟล์แต่ละนั้นทำให้การคิดว่าดึงข้อมูลโดยอิงจากบริษัทที่ดังและเป็นมาตรฐาน เป็นความคิดที่ผิดเนื่องจากผู้คนได้ใส่ทักษะความสามารถครอบคลุมเกินฟิลด์ที่ตัวเองทำงาน ดังนั้นในอนาคตจึงคิดเปลี่ยนการดึงข้อมูลโปรไฟล์จากบริษัท เปลี่ยนเป็นดึงผ่านฟิลด์ไอทีแทนซึ่งคาดว่าจะสามารถทำให้การวัดประสิทธิภาพครอบคลุมในส่วนของโปรไฟล์ด้วย