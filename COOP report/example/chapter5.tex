\chapter{สรุปผล}
\label{chapter:conclusion}
ในการทดลองนี้ ได้เสนอกรอบการทำงานของระบบแนะนำตำแหน่งงานโดยใช้เทคนิคการกรองแบบเนื้อหา โดยใช้ข้อมูลที่แตกต่างกันสองแหล่งคือ จ็อบเบสหรือข้อมูลตำแหน่งงานและยูสเซอร์เบสหรือข้อมูลโปรไฟล์ผู้ใช้มาใช้ในการการเทรนโมเดลโดยใช้เทคนิค "Support vector machine" ซึ่งให้ผลลัพธ์ที่แม่นยำที่สุดเมื่อเทียบกับเทคนิคอื่น ๆ ผลลัพธ์ของโมเดลการแบ่งกลุ่มสายจะได้ความแม่นยำอยู่ที่ 86\% เมื่อใช้ข้อมูลจากตำแหน่งาน และ 49\% เมื่อใช้ข้อมูลยูสเซอร์ โดยความแม่นยำทั้งสองนี้จะถูกถ่วงน้ำหนักและสเกลด้วย 100 เพื่อใช้ในการแบ่งข้อมูลในขั้นตอนการแนะนำงานโดยผลลัพธ์ของตำแหน่งงานที่แนะนำมามีความแม่นยำอยู่ในระดับที่น่าพึงพอใจเป็นอย่างมาก ถึงแม้ว่าชื่อและประเภทงานอาจมีความคลาดเคลื่อน แต่เนื้อหาของงานค่อนข้างตรงกับการจับคู่กับโปรไฟล์ผู้ใช้

\section{ปัญหาที่เกิดขึ้น}
ปัญหาที่เกิดขึ้นจากการทดลอง ผู้จัดทำได้พบว่าข้อมูลที่สกัดมาจากเว็บไซต์ลิงค์อินนั้น โปรไฟล์ผู้ใช้มักเขียนคำอธิบานตนเองค่อนข้างไม่เกี่ยวข้องกับลักษณะงานที่ทำ อีกทั้งเว็บไซต์ลิงค์อินมีความยากในการสกัดข้อมูลเป็นอย่างมาก ทำให้ข้อมูลที่ได้นั้นมีจำนวนยังไม่เพียงพอต่อการใช้งานในความเห็นของผู้จัดทำ โดยปริมาณโปรไฟล์ที่สกัดมานั้นมีจำนวน 2,720 คน จากที่คาดหวังไว้ 6,000+ \par
ปัญหาต่อมาที่พบคือเนื่องจากตำแหน่งงานในแต่ละรายการนั้นมีความไม่เหมือนใครในด้านเนื้อหางานถึงแม้ว่าหัวข้อจะเหมือนกันก็ตาม รวมถึงตำแหน่งงานมีเวลาหมดอายุหรือปิดรับสมัคร ทำให้การแนะนำรายการเดิมนั้นเป็นไปไม่ได้ ทำให้การแนะนำด้วยเทคนิคการกรองแบบร่วมกัน ไม่สามารถใช้ได้ \par

\section{ทิศทางในอนาคต}
ทิศทางในอนาคตผู้จัดทำมุ่งเน้นไปที่เรื่องของข้อมูลที่ได้มามากกว่าตัวโมเดล โดยระบบสกัดข้อมูลที่ใช้อยู่ตอนนี้ยังไม่มีความไม่สเถียรและต้องมีการปรับปรุงอีกมากในการสกัดข้อมูลจากทั้งสองแหล่ง ผู้จัดทำจึงมีแผนในการพัฒนาโครงสร้างท่อข้อมูลให้เป็นระบบที่เป็นระบบอัตโนมัติ เพื่อให้ได้ข้อมูลที่ใหม่อยู่เสมอและความแม่นยำที่แม่นขึ้นในการแบ่งกลุ่มสายงาน
