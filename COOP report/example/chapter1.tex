\chapter{บทนำ}
\label{chapter:introduction}

\section{ที่มาและความสำคัญ}

การหางานที่เหมาะสมกับตัวผู้หางานนั้น เป็นสิ่งที่เป็นปัญหามาอย่างช้านานจนถึงปัจจุบัน ไม่ว่าจะเป็นกลุ่มผู้เรียนจบใหม่ นักศึกษาฝึกงาน หรือแม้กระทั่งผู้ที่ต้องการเปลี่ยนงาน ปัญหานี้มีปัจจัยหลายส่วนและหลายกลุ่ม เช่น กลุ่มของผู้ฝึกงานและผู้ที่เรียนจบใหม่ มักยังไม่ทราบความต้องการของตนเองว่าต้องการทำงานในแขนงใหน ตนเองถนัดกับสิ่งใด และปัญหาภาพรวมที่พบเจอเยอะที่สุดคือความยุ่งยากในการหางาน โดยที่ผู้หางานจำเป็นต้องค้นหางานด้วยตนเองทีละตำแหน่งและอ่านรายละเอียดตำแหน่งเหล่านั้นว่ามีความต้องการตรงกับความสามารถเราหรือไม่ แต่เมื่อเลือกตำแหน่งงานได้ก็ไม่ได้หมายความว่างานเหล่านั้นจะเหมาะกับตัวผู้หางาน นั่นทำให้ต้องกลับมาค้นหาด้วยวิธีแบบเดิมอีกรอบ จากที่กล่าวมาจะเห็นว่าปัญหาเหล่านี้เป็นปัญหาที่สำคัญและยังเจออยู่ไม่ว่ายุคใหนก็ตาม

ทั้งนี้บางเว็บไซต์หางานก็ได้มีการแก้ปัญหาเหล่านี้ด้วยการเพิ่มระบบคัดกรองและแนะนำตำแหน่งงานขึ้นมา อย่างเช่นเว็บไซต์จ็อบบีเคเค (JobBKK) ที่มีระบบคัดกรองแบ่งเป็นประเภทที่ผู้หางานต้องการเช่น สถานที่ เงินเดือนขั้นต่ำ ประสบการณ์ อีกทั้งยังมีระบบจับคู่งานกับผู้หางาน แต่ถึงจะมีความละเอียดในการค้นหาและคัดกรอง แต่ต้องแลกมาด้วยความยุ่งยากและเสียเวลาเกินความจำเป็นในการค้นหางานแต่ละครั้ง อีกทั้งระบบจับคู่งานกับผู้หางานเป็นการจับคู่แค่ในหมวดหมู่งานนั้นเท่านั้น ไม่ได้จับคู่งานโดยอิงจากความสามารถจริง ๆ ของผู้หางานเป็นเคสต่อเคส

ด้วยปัญหาดังกล่าวและตัวอย่างเว็บหางานส่วนใหญ่ที่พบ ทางผู้จัดทำจึงได้คิดและออกแบบระบบที่สามารถจับคู่ทักษะวิชาชีพของผู้หางานกับตำแหน่งงานให้มีความสอดคล้องและมีประสิทธิภาพมากที่สุด โดยคำนึงการใช้งานได้จริง เพื่อช่วยแก้ปัญหาการหางานในปัจจุบัน ไม่ว่าจะเป็นการไม่ทราบความต้องการของตนเอง หรือความซับซ้อนและยุ่งยากในการหางาน โดยระบบที่ผู้จัดทำขึ้นมาคือระบบให้การแนะนำ (Recommendation System) เพื่อมาช่วยสนับสนุนเว็บแอพพลิเคชั่น (Web Application)จับคู่ผู้หางานกับตำแหน่งงาน ซึ่งเทคนิคที่ใช้ในการสร้างระบบแนะนำนั้น ทางผู้จัดทำได้ใช้เทคนิคการกรองแบบอิงเนื้อหา (Content Based Filtering) ซึ่งเป็นการแนะนำโดยทำการดูเนื้อหาและลักษณะของงานว่ามีคำสำคัญ (Keyword) และแนะนำงานที่มีลักษณะคล้ายกับโปรไฟล์ของผู้หางานมากที่สุดโดยคำนึงนึงทักษะวิชาชีพของผู้หางานเป็นหลัก ในการหาคำสำคัญของงานได้ใช้เทคนิคการประมวลผลภาษาธรรมชาติ (Natural Language Processing) เข้ามาช่วยในการเข้าใจและแบ่งคำเพื่อนำไปวิเคราะห์หาคำสำคัญต่อไป

\section{วัตถุประสงค์}
\begin{enumerate}
  \item เพื่ออกกแบบและพัฒนาออกแบบโครงสร้างท่อส่งข้อมูลอัตโนมัติ เพื่อจัดการการไหลของข้อมูลที่มาจากการสกัดและจัดการกับข้อมูลเหล่านี้ให้อยู่ในรูปแบบที่เป็นโครงสร้าง
  \item เพื่อออกแบบและพัฒนาระบบแนะนำตำแหน่งงานโดยผสมผสานการอ้างอิงจากทักษะของโปรไฟล์ยูสเซอร์ และระหว่างยูสเซอร์กับตำแหน่งงาน
  \item ประยุกต์ระบบแนะนำตำแหน่งงานกับเว็บแอพพลิเคชั่น ที่จะเปิดให้ใช้สำหรับหางานจากตำแหน่งงานที่สกัดมาจากเว็บไซต์ชั้นนำ
  \item ออกแบบและพัฒนาเว็บแอพพลิเคชั่นหางานที่สามารถจับคู่โปรไฟล์ยูสเซอร์กับตำแหน่งงานได้
\end{enumerate}

\section{ขอบเขตของงานวิจัย}
\begin{enumerate}
    \item พัฒนาระบบแนะนำตำแหน่งงานโดยใช้การกรองแบบอิงเนื้อหา (Content Based Filtering) โดยผสมระหว่างยูสเซอร์เบสและจ็อบเบส
    \item พัฒนาเว็บแอพพลิเคชั่นหางาน ที่เชื่อมต่อกับระบบแนะนำตำแหน่งงาน
    \item ข้อมูลที่ใช้ในการสร้างระบบในส่วนของข้อมูลตั้งต้น โปรไฟล์ได้ใช้ข้อมูลจาก ลิงกต์อิน (Linkedin) และส่วนของตำแหน่งงานได้ใช้ข้อมูลจาก อินดีด (Indeed)
    \item ข้อมูลที่สกัดและการสร้างโมเดลรวมถึงระบบจะเป็นภาษาอังกฤษทั้งหมด
    \item ขอบเขตของการแนะนำตำแหน่งงานจะอยู่ในขอบเขตของไอที
\end{enumerate}

\section{ประโยชน์ที่คาดว่าจะได้รับ}
ทางเราได้สร้างระบบแนะนำตำแหน่งงานขึ้นมาเพื่อช่วยในลดปัญหาความยุ่งยากซับซ้อนและเสียเวลากับวิธีหางานในปัจจุบัน โดยนำเสนอการจับคู่ระหว่างโปรไฟล์และตำแหน่งงานที่เหมาะสมกัน
\begin{enumerate}
  \item ช่วยให้ผู้ที่ยังไม่มีงานทำในปัจจุบันสามารถหางานได้ขึ้นผ่านขั้นตอนการจับคู่ตำแหน่งงานที่ทางเราสร้างขึ้น
  \item เพื่อสร้างฐานข้อมูลที่เป็นอนาคตในการนำข้อมูลตำแหน่งงานไปวิเคราะห์และใช้งานในอนาคต
  \item เพื่อวิจัยและค้นคว้าการสร้างระบบแนะนำทีมีประสิทธิภาพและต่อยอดได้ในอนาคต
\end{enumerate}


